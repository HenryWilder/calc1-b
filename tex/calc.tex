%%%%%%%%%%%%%%%%%%%%%%%%%%%%% Define Article %%%%%%%%%%%%%%%%%%%%%%%%%%%%%%%%%%
\documentclass{report}
%%%%%%%%%%%%%%%%%%%%%%%%%%%%%%%%%%%%%%%%%%%%%%%%%%%%%%%%%%%%%%%%%%%%%%%%%%%%%%%

%%%%%%%%%%%%%%%%%%%%%%%%%%%%% Using Packages %%%%%%%%%%%%%%%%%%%%%%%%%%%%%%%%%%
\usepackage{geometry}
\usepackage{graphicx}
\usepackage{amssymb}
\usepackage{amsmath}
\usepackage{amsthm}
\usepackage{empheq}
\usepackage{mdframed}
\usepackage{booktabs}
\usepackage{lipsum}
\usepackage{graphicx}
\usepackage{color}
\usepackage{transparent}
\usepackage{psfrag}
\usepackage{pgfplots}
\usepackage{bm}
\usepackage{xparse}
%%%%%%%%%%%%%%%%%%%%%%%%%%%%%%%%%%%%%%%%%%%%%%%%%%%%%%%%%%%%%%%%%%%%%%%%%%%%%%%

\makeatletter
\DeclareRobustCommand{\lnor}{
    \mathbin{\overline{\vee}}
}
\DeclareRobustCommand{\lxnor}{
    \mathbin{\underline{\wedge}}
}
\makeatother

\def\undefined{\mathrm{undefined}}
\def\dne{\mathrm{DNE}}

\def\lnand{\barwedge}
\def\lxor{\veebar}
\def\true{1} % \top or 1
\def\false{0} % \bot or 0

% Options:
% - leibniz
% - lagrange
% - euler
\newcommand{\CalcNotation}{lagrange}

\NewDocumentCommand{\deriv}{ s e{^} m d() g d() }{
    % #1: Put in top of Leibniz fraction
    % #2: Order of derivative
    % #3: With respect to
    % #4: Expression to take derivative of (in parentheses)
    % #5: Expression to take derivative of (no parentheses)
    % #6: Argument(s) of expression
    \ifdefstring{\CalcNotation}{leibniz}{
        % Leibniz
        \fn{\frac{
            \IfValueTF{#2}{ d^{#2} }{ d }
            \IfBooleanT{#1}{ \IfValueTF{#4}{ \group(#4) }{ #5 } }
        }{
            \IfValueTF{#2}{ d{#3}^{#2} }{ d{#3} }
        }}
        \IfBooleanF{#1}{
            \IfValueTF{#4}{ \group(#4) }{ #5 }
        }
        \IfValueT{#6}{ \group(#6) }
    }{\ifdefstring{\CalcNotation}{lagrange}{
        % Lagrange (prime)
        \IfValueTF{#4}{ \group(#4) }{ #5 }
        ^{
            \fn{
                \IfValueTF{#2}{
                    \ifnumequal{#2}{1}{
                        \prime
                    }{\ifnumequal{#2}{2}{
                        \prime\prime
                    }{
                        \left(#2\right)
                    }}
                }{
                    \fn{\prime}
                }
            }
        }
        \IfValueT{#6}{ \group(#6) }
    }{\ifdefstring{\CalcNotation}{euler}{
        % Euler
        \fn{\IfValueTF{#2}{ D^{#2}_{#3} }{ D_{#3} }}
        \IfValueTF{#4}{ \group(#4) }{ #5 }
        \IfValueT{#6}{ \group(#6) }
    }{
        \errmessage{I don't know how to represent ``\CalcNotation'' notation.}
    }}}
    % Validation
    \IfValueTF{#4}{
        \IfValueT{#5}{
            \errmessage{
                Derivative expression cannot simultaneously have and lack parentheses.
                Consider enclosing the command in braces if either ``(#4)'' or ``{#5}'' is not intended as an argument
            }
        }
    }{
        \IfValueF{#5}{
            \errmessage{Must provide an expression to take the derivative of.}
        }
    }
}

\def\ColorReset{\color[HTML]{cccccc}}

\newcounter{brDepth}
\NewDocumentCommand{\brColor}{}{
    \color[HTML]{
        \ifnumequal{\value{brDepth}}{1}{
            ffd700
        }{\ifnumequal{\value{brDepth}}{2}{
            da70d6
        }{\ifnumequal{\value{brDepth}}{3}{
            179fff
        }{
            \errmessage{unreachable}
        }}}
    }
}
\NewDocumentCommand{\brPush}{}{
    \stepcounter{brDepth}
    \ifnumequal{\value{brDepth}}{4}{
        \setcounter{brDepth}{1}
    }{}
}
\NewDocumentCommand{\brPop}{}{
    \addtocounter{brDepth}{-1}
    \ifnumequal{\value{brDepth}}{0}{
        \setcounter{brDepth}{3}
    }{}
}
\NewDocumentCommand{\br}{mmm}{
    \brPush
    {\brColor
        \left#1
        {\ColorReset #2 }
        \right#3
    }
    \brPop
}
\NewDocumentCommand{\group}{r()}{
    \br({#1})
}

\NewDocumentCommand{\type}{m}{
    {\color[HTML]{4ec9b0}#1}
}
\NewDocumentCommand{\lit}{m}{
    {\color[HTML]{b5cea8}#1}
}
\NewDocumentCommand{\keyword}{m}{
    {\color[HTML]{569cd6}#1}
}
\NewDocumentCommand{\op}{m}{
    \mathbin{\color[HTML]{569cd6}#1}
}
\NewDocumentCommand{\stmt}{m}{
    \mathrel{\color[HTML]{c586c0}#1}
}
\NewDocumentCommand{\aside}{m}{
    \mathrlap{\qquad\color[HTML]{6a9955}#1}
}
\NewDocumentCommand{\var}{d<>m}{
    {\color[HTML]{9cdcfe}#2}
}
\NewDocumentCommand{\const}{d<>m}{
    {\color[HTML]{4fc1ff}#2}
}
\NewDocumentCommand{\ghost}{m}{
    {\transparent{0.34}\color{white}#1}
}

\def\R{{\type{\mathbb{R}}}}
\def\N{{\type{\mathbb{N}}}}
\def\Z{{\type{\mathbb{Z}}}}
\NewDocumentCommand{\fn}{md()}{
    {\color[HTML]{dcdcaa}
        #1
        \IfValueT{#2}{
            \group(#2)
        }
    }
}

\newbool{UseBinomials}
\booltrue{UseBinomials}

\NewCommandCopy{\builtinBinom}{\binom}
\RenewDocumentCommand{\binom}{mm}{
    \ifbool{UseBinomials}{
        \brPush
        {\brColor
            \builtinBinom{
                {\ColorReset #1}
            }{
                {\ColorReset #2}
            }
        }
        \brPop
    }{
        \op{\frac{
            \var{n}\op!
        }{
            \var{i}\op! \mul \br({\var{n} \op- \var{i}})\op!
        }}
    }
}
\NewCommandCopy{\builtinLog}{\log}
\RenewDocumentCommand{\log}{e{_}g}{
    \fn{\builtinLog_{#1}}({#2})
}
\NewCommandCopy{\builtinLn}{\ln}
\RenewDocumentCommand{\ln}{g}{
    \fn{\builtinLn}({#1})
}
\NewCommandCopy{\builtinFrac}{\frac}
\RenewDocumentCommand{\frac}{mm}{
    \op{\builtinFrac{#1}{#2}}
}
\NewCommandCopy{\builtinSqrt}{\sqrt}
\RenewDocumentCommand{\sqrt}{om}{
    \op{\IfValueTF{#1}{\builtinSqrt[#1]{#2}}{\builtinSqrt{#2}}}
}
\NewDocumentCommand{\mul}{mm}{
    #1 \op\cdot #2
}

%%%%%%%%%%%%%%%%%%%%%%%%%% Page Setting %%%%%%%%%%%%%%%%%%%%%%%%%%%%%%%%%%%%%%%
\geometry{a4paper}
\pagecolor[RGB]{31,31,31}
% \pagecolor[RGB]{0,0,0}
\ColorReset
\addtolength{\jot}{2ex}

%%%%%%%%%%%%%%%%%%%%%%%%%% Define some useful colors %%%%%%%%%%%%%%%%%%%%%%%%%%
\definecolor{ocre}{RGB}{243,102,25}
\definecolor{mygray}{RGB}{243,243,244}
\definecolor{deepGreen}{RGB}{26,111,0}
\definecolor{shallowGreen}{RGB}{235,255,255}
\definecolor{deepBlue}{RGB}{61,124,222}
\definecolor{shallowBlue}{RGB}{235,249,255}
%%%%%%%%%%%%%%%%%%%%%%%%%%%%%%%%%%%%%%%%%%%%%%%%%%%%%%%%%%%%%%%%%%%%%%%%%%%%%%%

%%%%%%%%%%%%%%%%%%%%%%%%%% Define an orangebox command %%%%%%%%%%%%%%%%%%%%%%%%
\newcommand\orangebox[1]{\fcolorbox{ocre}{mygray}{\hspace{1em}#1\hspace{1em}}}
%%%%%%%%%%%%%%%%%%%%%%%%%%%%%%%%%%%%%%%%%%%%%%%%%%%%%%%%%%%%%%%%%%%%%%%%%%%%%%%

%%%%%%%%%%%%%%%%%%%%%%%%%%%% English Environments %%%%%%%%%%%%%%%%%%%%%%%%%%%%%
\newtheoremstyle{mytheoremstyle}{3pt}{3pt}{\normalfont}{0cm}{\rmfamily\bfseries}{}{1em}{{\color{black}\thmname{#1}~\thmnumber{#2}}\thmnote{\,--\,#3}}
\newtheoremstyle{myproblemstyle}{3pt}{3pt}{\normalfont}{0cm}{\rmfamily\bfseries}{}{1em}{{\color{black}\thmname{#1}~\thmnumber{#2}}\thmnote{\,--\,#3}}
\theoremstyle{mytheoremstyle}
\newmdtheoremenv[linewidth=1pt,backgroundcolor=shallowGreen,linecolor=deepGreen,leftmargin=0pt,innerleftmargin=20pt,innerrightmargin=20pt,]{theorem}{Theorem}[section]
\theoremstyle{mytheoremstyle}
\newmdtheoremenv[linewidth=1pt,backgroundcolor=shallowBlue,linecolor=deepBlue,leftmargin=0pt,innerleftmargin=20pt,innerrightmargin=20pt,]{definition}{Definition}[section]
\theoremstyle{myproblemstyle}
\newmdtheoremenv[linecolor=black,leftmargin=0pt,innerleftmargin=10pt,innerrightmargin=10pt,]{problem}{Problem}[section]
%%%%%%%%%%%%%%%%%%%%%%%%%%%%%%%%%%%%%%%%%%%%%%%%%%%%%%%%%%%%%%%%%%%%%%%%%%%%%%%

%%%%%%%%%%%%%%%%%%%%%%%%%%%%%%% Plotting Settings %%%%%%%%%%%%%%%%%%%%%%%%%%%%%
\usepgfplotslibrary{colorbrewer}
\pgfplotsset{width=8cm,compat=1.9}
%%%%%%%%%%%%%%%%%%%%%%%%%%%%%%%%%%%%%%%%%%%%%%%%%%%%%%%%%%%%%%%%%%%%%%%%%%%%%%%

%%%%%%%%%%%%%%%%%%%%%%%%%%%%%%% Title & Author %%%%%%%%%%%%%%%%%%%%%%%%%%%%%%%%
\title{Math Properties}
\author{Amy Wilder}
%%%%%%%%%%%%%%%%%%%%%%%%%%%%%%%%%%%%%%%%%%%%%%%%%%%%%%%%%%%%%%%%%%%%%%%%%%%%%%%

\begin{document}
    \maketitle

    \chapter{Properties}

    % Derivative unit tests
    % \clearpage
    % \begin{gather*}
    %     \def\CalcNotation{leibniz}
    %     \begin{matrix*}
    %         \displaystyle \deriv        {\var{x}}          {\fn{f}}           &
    %         \displaystyle \deriv *      {\var{x}}          {\fn{f}}           &
    %         \displaystyle \deriv        {\var{x}} (\fn{f})                    &
    %         \displaystyle \deriv *      {\var{x}} (\fn{f})                    &
    %         \displaystyle \deriv        {\var{x}}          {\fn{f}} (\var{x}) &
    %         \displaystyle \deriv *      {\var{x}}          {\fn{f}} (\var{x}) &
    %         \displaystyle \deriv        {\var{x}} (\fn{f})          (\var{x}) &
    %         \displaystyle \deriv *      {\var{x}} (\fn{f})          (\var{x}) \\[2em]
    %         \displaystyle \deriv   ^{1} {\var{x}}          {\fn{f}}           &
    %         \displaystyle \deriv * ^{1} {\var{x}}          {\fn{f}}           &
    %         \displaystyle \deriv   ^{1} {\var{x}} (\fn{f})                    &
    %         \displaystyle \deriv * ^{1} {\var{x}} (\fn{f})                    &
    %         \displaystyle \deriv   ^{1} {\var{x}}          {\fn{f}} (\var{x}) &
    %         \displaystyle \deriv * ^{1} {\var{x}}          {\fn{f}} (\var{x}) &
    %         \displaystyle \deriv   ^{1} {\var{x}} (\fn{f})          (\var{x}) &
    %         \displaystyle \deriv * ^{1} {\var{x}} (\fn{f})          (\var{x}) \\[2em]
    %         \displaystyle \deriv   ^{2} {\var{x}}          {\fn{f}}           &
    %         \displaystyle \deriv * ^{2} {\var{x}}          {\fn{f}}           &
    %         \displaystyle \deriv   ^{2} {\var{x}} (\fn{f})                    &
    %         \displaystyle \deriv * ^{2} {\var{x}} (\fn{f})                    &
    %         \displaystyle \deriv   ^{2} {\var{x}}          {\fn{f}} (\var{x}) &
    %         \displaystyle \deriv * ^{2} {\var{x}}          {\fn{f}} (\var{x}) &
    %         \displaystyle \deriv   ^{2} {\var{x}} (\fn{f})          (\var{x}) &
    %         \displaystyle \deriv * ^{2} {\var{x}} (\fn{f})          (\var{x}) \\[2em]
    %         \displaystyle \deriv   ^{3} {\var{x}}          {\fn{f}}           &
    %         \displaystyle \deriv * ^{3} {\var{x}}          {\fn{f}}           &
    %         \displaystyle \deriv   ^{3} {\var{x}} (\fn{f})                    &
    %         \displaystyle \deriv * ^{3} {\var{x}} (\fn{f})                    &
    %         \displaystyle \deriv   ^{3} {\var{x}}          {\fn{f}} (\var{x}) &
    %         \displaystyle \deriv * ^{3} {\var{x}}          {\fn{f}} (\var{x}) &
    %         \displaystyle \deriv   ^{3} {\var{x}} (\fn{f})          (\var{x}) &
    %         \displaystyle \deriv * ^{3} {\var{x}} (\fn{f})          (\var{x})
    %     \end{matrix*}
    %     \\[2em]
    %     \def\CalcNotation{lagrange}
    %     \begin{matrix*}
    %         \displaystyle \deriv        {\var{x}}          {\fn{f}}           &
    %         \displaystyle \deriv *      {\var{x}}          {\fn{f}}           &
    %         \displaystyle \deriv        {\var{x}} (\fn{f})                    &
    %         \displaystyle \deriv *      {\var{x}} (\fn{f})                    &
    %         \displaystyle \deriv        {\var{x}}          {\fn{f}} (\var{x}) &
    %         \displaystyle \deriv *      {\var{x}}          {\fn{f}} (\var{x}) &
    %         \displaystyle \deriv        {\var{x}} (\fn{f})          (\var{x}) &
    %         \displaystyle \deriv *      {\var{x}} (\fn{f})          (\var{x}) \\[2em]
    %         \displaystyle \deriv   ^{1} {\var{x}}          {\fn{f}}           &
    %         \displaystyle \deriv * ^{1} {\var{x}}          {\fn{f}}           &
    %         \displaystyle \deriv   ^{1} {\var{x}} (\fn{f})                    &
    %         \displaystyle \deriv * ^{1} {\var{x}} (\fn{f})                    &
    %         \displaystyle \deriv   ^{1} {\var{x}}          {\fn{f}} (\var{x}) &
    %         \displaystyle \deriv * ^{1} {\var{x}}          {\fn{f}} (\var{x}) &
    %         \displaystyle \deriv   ^{1} {\var{x}} (\fn{f})          (\var{x}) &
    %         \displaystyle \deriv * ^{1} {\var{x}} (\fn{f})          (\var{x}) \\[2em]
    %         \displaystyle \deriv   ^{2} {\var{x}}          {\fn{f}}           &
    %         \displaystyle \deriv * ^{2} {\var{x}}          {\fn{f}}           &
    %         \displaystyle \deriv   ^{2} {\var{x}} (\fn{f})                    &
    %         \displaystyle \deriv * ^{2} {\var{x}} (\fn{f})                    &
    %         \displaystyle \deriv   ^{2} {\var{x}}          {\fn{f}} (\var{x}) &
    %         \displaystyle \deriv * ^{2} {\var{x}}          {\fn{f}} (\var{x}) &
    %         \displaystyle \deriv   ^{2} {\var{x}} (\fn{f})          (\var{x}) &
    %         \displaystyle \deriv * ^{2} {\var{x}} (\fn{f})          (\var{x}) \\[2em]
    %         \displaystyle \deriv   ^{3} {\var{x}}          {\fn{f}}           &
    %         \displaystyle \deriv * ^{3} {\var{x}}          {\fn{f}}           &
    %         \displaystyle \deriv   ^{3} {\var{x}} (\fn{f})                    &
    %         \displaystyle \deriv * ^{3} {\var{x}} (\fn{f})                    &
    %         \displaystyle \deriv   ^{3} {\var{x}}          {\fn{f}} (\var{x}) &
    %         \displaystyle \deriv * ^{3} {\var{x}}          {\fn{f}} (\var{x}) &
    %         \displaystyle \deriv   ^{3} {\var{x}} (\fn{f})          (\var{x}) &
    %         \displaystyle \deriv * ^{3} {\var{x}} (\fn{f})          (\var{x})
    %     \end{matrix*}
    %     \\[2em]
    %     \def\CalcNotation{euler}
    %     \begin{matrix*}
    %         \displaystyle \deriv        {\var{x}}          {\fn{f}}           &
    %         \displaystyle \deriv *      {\var{x}}          {\fn{f}}           &
    %         \displaystyle \deriv        {\var{x}} (\fn{f})                    &
    %         \displaystyle \deriv *      {\var{x}} (\fn{f})                    &
    %         \displaystyle \deriv        {\var{x}}          {\fn{f}} (\var{x}) &
    %         \displaystyle \deriv *      {\var{x}}          {\fn{f}} (\var{x}) &
    %         \displaystyle \deriv        {\var{x}} (\fn{f})          (\var{x}) &
    %         \displaystyle \deriv *      {\var{x}} (\fn{f})          (\var{x}) \\[2em]
    %         \displaystyle \deriv   ^{1} {\var{x}}          {\fn{f}}           &
    %         \displaystyle \deriv * ^{1} {\var{x}}          {\fn{f}}           &
    %         \displaystyle \deriv   ^{1} {\var{x}} (\fn{f})                    &
    %         \displaystyle \deriv * ^{1} {\var{x}} (\fn{f})                    &
    %         \displaystyle \deriv   ^{1} {\var{x}}          {\fn{f}} (\var{x}) &
    %         \displaystyle \deriv * ^{1} {\var{x}}          {\fn{f}} (\var{x}) &
    %         \displaystyle \deriv   ^{1} {\var{x}} (\fn{f})          (\var{x}) &
    %         \displaystyle \deriv * ^{1} {\var{x}} (\fn{f})          (\var{x}) \\[2em]
    %         \displaystyle \deriv   ^{2} {\var{x}}          {\fn{f}}           &
    %         \displaystyle \deriv * ^{2} {\var{x}}          {\fn{f}}           &
    %         \displaystyle \deriv   ^{2} {\var{x}} (\fn{f})                    &
    %         \displaystyle \deriv * ^{2} {\var{x}} (\fn{f})                    &
    %         \displaystyle \deriv   ^{2} {\var{x}}          {\fn{f}} (\var{x}) &
    %         \displaystyle \deriv * ^{2} {\var{x}}          {\fn{f}} (\var{x}) &
    %         \displaystyle \deriv   ^{2} {\var{x}} (\fn{f})          (\var{x}) &
    %         \displaystyle \deriv * ^{2} {\var{x}} (\fn{f})          (\var{x}) \\[2em]
    %         \displaystyle \deriv   ^{3} {\var{x}}          {\fn{f}}           &
    %         \displaystyle \deriv * ^{3} {\var{x}}          {\fn{f}}           &
    %         \displaystyle \deriv   ^{3} {\var{x}} (\fn{f})                    &
    %         \displaystyle \deriv * ^{3} {\var{x}} (\fn{f})                    &
    %         \displaystyle \deriv   ^{3} {\var{x}}          {\fn{f}} (\var{x}) &
    %         \displaystyle \deriv * ^{3} {\var{x}}          {\fn{f}} (\var{x}) &
    %         \displaystyle \deriv   ^{3} {\var{x}} (\fn{f})          (\var{x}) &
    %         \displaystyle \deriv * ^{3} {\var{x}} (\fn{f})          (\var{x})
    %     \end{matrix*}
    % \end{gather*}
    % \clearpage

    % Assume the following:
    % \begin{itemize}
    %     \item \(\const{h},\const{k}\) are constants
    %     \item \(\var{\theta},\var{\phi},\var{\psi}\) are angles (the golden ratio is \(\lit{\varphi}\))
    %     \item \(\var{a},\var{b},\var{c},\var{x} \in \R\)
    %     \item \(\var{y}\) is defined in terms of \(\var{x}\)
    %     \item \(\fn{f}\) and \(\fn{g}\) are functions
    %     \item \(\var<\N>{n},\var<\N>{m} \in \N\)
    % \end{itemize}

    \section{Reflexive}
    \begin{gather}
        \var{x} \stmt= \var{x}
    \end{gather}

    \section{Symmetric}
    \begin{align}
        \var{a} \stmt= \var{b} &\stmt{\iff} \var{b} \stmt= \var{a} \\
        \var{a} \stmt\ne \var{b} &\stmt{\iff} \var{b} \stmt\ne \var{a} \\
        \var{a} \stmt< \var{b} &\stmt{\iff} \var{b} \stmt> \var{a} \\
        \var{a} \stmt\le \var{b} &\stmt{\iff} \var{b} \stmt\ge \var{a}
    \end{align}

    \section{Transitive}
    \begin{align}
        \br({\var{a} \stmt= \var{b} \stmt\bigwedge \var{b} \stmt= \var{c}}) &\stmt\implies \var{a} \stmt= \var{c}
    \end{align}

    \section{Substitution}
    \begin{align}
        \var{a} \stmt= \var{b} &\stmt\iff \fn{f}(\var{a}) \stmt= \fn{f}(\var{b})
    \end{align}

    \section{Identity}
    \begin{align}
        \var{x} \op+ \lit{0} &\stmt= \var{x} \\
        \var{x} \op- \lit{0} &\stmt= \var{x} \\
        \mul{\var{x}}{\lit{1}} &\stmt= \var{x} \\
        \frac{\var{x}}{\lit{1}} &\stmt= \var{x} \\
        \var{x}^\lit{1} &\stmt= \var{x} \\
        \deriv{\var{x}}(\lit{e}^{\var{x}}) &\stmt= \lit{e}^{\var{x}}
    \end{align}

    \section{Fixed Point}
    \begin{align}
        \mul{\var{x}}{\lit{0}} &\stmt= \lit{0} \\
        {\var{x}}^\lit{0} &\stmt= \lit{1} \\
        \log_{\var{x}}{\lit{1}} &\stmt= \lit{0} \\
        \deriv*{\var{x}}{\var{x}} &\stmt= \lit{1}
    \end{align}

    \section{Cancellation}
    \begin{align}
        \var{x} \op- \var{x} &\stmt= \lit{0} \\
        \frac{\var{x}}{\var<\ne0>{x}} &\stmt= \lit{1} \\
        \log_{\var{x}}{\var{x}} &\stmt= \lit{1} \\
        \ln{\lit{e}} &\stmt= \lit{1} \\
        \deriv*{\var{x}}{\const{c}} &\stmt= \lit{0}
    \end{align}

    \section{Commutative}
    \begin{align}
        \var{a} \op+ \var{b} &\stmt= \var{b} \op+ \var{a} \\
        \mul{\var{a}}{\var{b}} &\stmt= \mul{\var{b}}{\var{a}}
    \end{align}

    \section{Associative}
    \begin{align}
        \group(\var{a} \op+ \var{b}) \op+ \var{c}
            &\stmt=
            \var{a} \op+ \group(\var{b} \op+ \var{c})
        \\
        \mul{\group(\mul{\var{a}}{\var{b}})}{\var{c}}
            &\stmt=
            \mul{\var{a}}{\group(\mul{\var{b}}{\var{c}})}
    \end{align}

    \section{Sum of}
    \begin{align}
        \mul{\var{a}}{\var{c}} \op+ \mul{\var{b}}{\var{c}}
            &\stmt=
            \mul{\group(\var{a} \op+ \var{b})}{\var{c}}
        && \text{Products} \label{sumof-mul}
        \\
        \log_{\var{n}}{\var{a}} \op+ \log_{\var{n}}{\var{b}}
            &\stmt=
            \log_{\var{n}}{\mul{\var{a}}{\var{b}}}
        && \text{Logarithms} \label{sumof-log}
        \\
        \deriv{\var{x}}{\br({\fn{f}(\var{x}) \op+ \fn{g}(\var{x})})}
            &\stmt=
            \deriv{\var{x}}{\fn{f}}(\var{x}) \op+ \deriv{\var{x}}{\fn{g}}(\var{x})
        && \text{Derivatives} \label{sumof-deriv}
    \end{align}

    \section{Difference of}
    \begin{align}
        \log_{\var{n}}{\var{a}} \op- \log_{\var{n}}{\var{b}}
            &\stmt=
            \log_{\var{n}}{\frac{\var{a}}{\var{b}}}
        && \text{Logarithms} \label{differenceof-log}
        \\
        \deriv{\var{x}}{\br({\fn{f}(\var{x}) \op- \fn{g}(\var{x})})}
            &\stmt=
            \deriv{\var{x}}{\fn{f}}(\var{x}) \op- \deriv{\var{x}}{\fn{g}}(\var{x})
        && \text{Derivatives} \label{differenceof-deriv}
    \end{align}

    \section{Product of}
    \begin{align}
        \mul{\var{x}^{\var{n}}}{\var{x}^{\var{m}}}
            &\stmt=
            \var{x}^{\var{n} \op+ \var{m}}
        && \text{Powers} \label{productof-pow}
        \\
        \mul{\var{m}}{\log_{\var{n}}{\var{x}}}
            &\stmt=
            \log_{\var{n}}{\var{x}^{\var{m}}}
        && \text{Logarithms} \label{productof-log-const}
        \\
        \deriv{\var{x}}(\mul{\fn{f}(\var{x})}{\fn{g}(\var{x})})
            &\stmt=
            \mul{\deriv{\var{x}}{\fn{f}}(\var{x})}{\fn{g}(\var{x})}
            \op+
            \mul{\fn{g}(\var{x})}{\deriv{\var{x}}{\fn{g}}(\var{x})}
        && \text{Derivatives} \label{productof-deriv}
    \end{align}

    \section{Quotient of}
    \begin{align}
        \frac{\var{x}^{\var{n}}}{\var{x}^{\var{m}}}
            &\stmt=
            \var{x}^{\var{n} \op- \var{m}}
        && \text{Powers} \label{quotientof-pow}
        \\
        \frac{\log_{\var{n}}{\var{x}}}{\log_{\var{n}}{\var{y}}}
            &\stmt=
            \log_{\var{y}}{\var{x}}
        && \text{Logarithms} \label{quotientof-log}
        \\
        \deriv{\var{x}}{\group(\frac{\fn{f}(\var{x})}{\fn{g}(\var{x})})}
            &\stmt=
            \frac{
                \mul{\deriv{\var{x}}{\fn{f}}(\var{x})}{\fn{g}(\var{x})}
                \op+
                \mul{\fn{g}(\var{x})}{\deriv{\var{x}}{\fn{g}}(\var{x})}
            }{
                \fn{g^{\lit{2}}}(\var{x})
            }
        && \text{Derivatives} \label{quotientof-deriv}
    \end{align}

    \section{Power of}
    \begin{align}
        {\group(\var{x}^{\var{n}})}^{\var{m}}
            &\stmt=
            \var{x}^{\mul{\var{n}}{\var{m}}}
        && \text{Powers} \label{powerof-pow}
        \\
        {\group(\mul{\var{x}}{\var{y}})}^{\var{n}}
            &\stmt=
            \mul{\var{x}^{\var{n}}}{\var{y}^{\var{n}}}
        && \text{Products} \label{powerof-mul}
        \\
        {\group(\var{a} \op+ \var{b})}^{\var{n}}
            &\stmt=
            \fn{\sum_{\var{i}=\lit{0}}^{\var{n}}}(
                \mul{
                    \binom{\var{n}}{\var{i}}
                }{\mul{
                    \var{a}^{\var{n} \op- \var{i}}
                }{
                    \var{b}^{\var{i}}
                }}
            )
        && \text{Sums} \label{powerof-sum}
        \\
        {\group(\frac{\var{x}}{\var{y}})}^{\var{n}}
            &\stmt=
            \frac{\var{x}^{\var{n}}}{\var{y}^{\var{n}}}
        && \text{Quotients} \label{powerof-div}
        \\
        \deriv{\var{x}}{\group(\var{x}^{\var{n}})}
            &\stmt=
            \mul{\var{n}}{\var{x}^{\var{n} \op- \lit{1}}}
        && \text{Derivatives} \label{powerof-deriv}
    \end{align}

    \section{Log of}
    \begin{align}
        \deriv{\var{x}}{\log_{\var{n}}}(\var{x})
            &\stmt=
            \frac{\lit{1}}{\mul{\var{x}}{\fn{\ln}(\var{n})}}
        && \text{Derivatives} \label{logof-deriv}
    \end{align}
    \subsection{Natural Log of}
    \begin{align}
        \deriv{\var{x}}{\ln}(\var{x})
            &\stmt=
            \frac{\lit{1}}{\var{x}}
        && \text{Derivatives} \label{lnof-deriv}
    \end{align}

    \section{Exponent of}
    \begin{align}
        \var{x}^{\op{\var{n}/\var{m}}}
            &\stmt=
            \sqrt[\var{m}]{\var{x}^{\var{n}}}
        && \text{Quotients} \label{exponentof-div}
        \\
        \var{x}^{\op-\var{n}}
            &\stmt=
            \frac{\lit{1}}{\var{x}^{\var{n}}}
        && \text{Negation} \label{exponentof-neg}
        \\
        \deriv{\var{x}}{\group(\var{a}^{\var{x}})}
            &\stmt=
            \mul{
                \var{a}^{\var{x}}
            }{
                \fn{\ln}(\var{a})
            }
        && \text{Derivatives} \label{exponentof-deriv}
    \end{align}

    \section{Chain}
    \begin{align}
        \deriv{\var{x}}(\fn{f} \op\circ \fn{g})(\var{x})
            &\stmt=
            \mul{
                \deriv{\var{x}}{\fn{f}}(\fn{g}(\var{x}))
            }{
                \deriv{\var{x}}{\fn{g}}(\var{x})
            }
        \label{chainof-deriv}
    \end{align}

    %%%%%%%%%%%%%%%%%%%%%%%%%%%%%%%%%%%%%%%%%%%%%%%%%%%%%%

    % \chapter{Operations}

    % \section{Arithmetic}
    % \begin{align}
    %     a + b &= b + a \label{eq:commut-add} \\
    %     a \cdot b &= b \cdot a \label{eq:commut-mult} \\
    %     (a + b) + c &= a + (b + c) = a + b + c \label{eq:assoc-add} \\
    %     (a \cdot b) \cdot c &= a \cdot (b \cdot c) = a \cdot b \cdot c \label{eq:assoc-mul} \\
    %     a \cdot (b + c) &= a \cdot b + a \cdot c \label{distrib-mul} \\
    %     \frac{a + b}{c} &= \frac{a}{c} + \frac{b}{c} \label{distrib-div}
    % \end{align}

    % \subsection{Logarithms}
    % \begin{align*}
    %     \log_a b = y &\iff a^y = b \\
    %     \log x &= \log_{10} x \\
    %     \ln x &= \log_{e} x \\
    % \end{align*}
    % \begin{align}
    %     \log_n\left( a \cdot b \right) &= \log_n a + \log_n b \\
    %     \log_n\left( \frac{a}{b} \right) &= \log_n a - \log_n b \\
    %     \log_n\left( a^b \right) &= b \cdot \log_n a\\
    %     \log_n 1 &= 0
    % \end{align}

    % \subsubsection{Natural}
    % \begin{align}
    %     \ln\left( a \cdot b \right) &= \ln a + \ln b \\
    %     \ln\left( \frac{a}{b} \right) &= \ln a - \ln b \\
    %     \ln\left( a^b \right) &= b\ln a \\
    %     \ln e &= 1
    % \end{align}

    % \section{Boolean}
    % \subsection{Definitions}
    % \begin{align*}
    %     \lnot  &= \text{NOT}  \\
    %     \land  &= \text{AND}  \\
    %     \lnand &= \text{NAND} \\
    %     \lor   &= \text{OR}   \\
    %     \lnor  &= \text{NOR}  \\
    %     \lxor  &= \text{XOR}  \\
    %     \lxnor &= \text{XNOR}
    % \end{align*}
    % \subsection{Using only NOT and OR}
    % \begin{align}
    %     a \lor b &= a \lor b \\
    %     a \lnor b &= \lnot(a \lor b) \\
    %     a \land b &= \lnot(\lnot a \lor \lnot b) \\
    %     a \lnand b &= \lnot a \lor \lnot b \\
    %     a \lxor b &= \lnot(\lnot (a \lor b) \lor \lnot(\lnot a \lor \lnot b)) \\
    %     a \lxnor b &= \lnot(a \lor b) \lor \lnot (\lnot a \lor \lnot b)
    % \end{align}
    % \subsection{Commutative}
    % \begin{align}
    %     a \lor b &= b \lor a \\
    %     a \lnor b &= b \lnor a \\
    %     a \land b &= b \land a \\
    %     a \lnand b &= b \lnand a \\
    %     a \lxor b &= b \lxor a \\
    %     a \lxnor b &= b \lxnor a \\
    % \end{align}
    % \subsection{Associative}
    % \begin{align}
    %     (a \land b) \land c &= a \land (b \land c) = a \land b \land c \\
    %     (a \lor b) \lor c &= a \lor (b \lor c) = a \lor b \lor c \\
    % \end{align}
    % \begin{align}
    %     % negation
    %     \lnot(\lnot x) &= x \\
    %     \lnot(a \land b) &= (\lnot a \lor \lnot b) \\
    %     \lnot(a \lor b) &= (\lnot a \land \lnot b) \\
    %     (a \land b) \lor (a \land c) &= a \land (b \lor c) \\
    %     (a \lor b) \land (a \lor c) &= a \lor (b \land c) \\
    %     x \land x &= x \\
    %     x \lor x &= x \\
    %     x \land \true &= x \\
    %     x \lor \false &= x \\
    %     x \land \false &= \false \\
    %     x \lor \true &= \true
    % \end{align}

    \section{Trig}
    \subsection{Definitions}
    \begin{align}
        \sin\theta &= \frac{y}{r} = \frac{1}{\csc\theta} \\
        \cos\theta &= \frac{x}{r} = \frac{1}{\sec\theta} \\
        \tan\theta &= \frac{y}{x} = \frac{1}{\cot\theta} = \frac{\sin\theta}{\cos\theta} \\
        \csc\theta &= \frac{r}{y} = \frac{1}{\sin\theta} \\
        \sec\theta &= \frac{r}{x} = \frac{1}{\cos\theta} \\
        \cot\theta &= \frac{x}{y} = \frac{1}{\tan\theta} = \frac{\cos\theta}{\sin\theta}
    \end{align}
    \subsection{Pythagorean}
    \begin{align}
        \cos^2\theta+\sin^2\theta &= 1 \\
        \sin^2\theta &= 1 - \cos^2 \theta \\
        \sin\theta &= \pm\sqrt{1 - \cos^2 \theta} \\
        \cos^2\theta &= 1 - \sin^2 \theta \\
        \cos\theta &= \pm\sqrt{1 - \sin^2 \theta} \\
    \end{align}
    \subsection{Angle Sum and Difference}
    \begin{align}
        \sin(\alpha \pm \beta) &= \sin\alpha\cdot\cos\beta \pm \cos\alpha\cdot\sin\beta \\
        \cos(\alpha \pm \beta) &= \cos\alpha\cdot\cos\beta \mp \sin\alpha\cdot\sin\beta \\
        \tan(\alpha \pm \beta) &= \frac{\tan\alpha \pm \tan\beta}{1 \mp \tan\alpha\cdot\tan\beta} \\
        \sin(2\theta) &= 2\sin\theta\cdot\cos\theta \\
        \cos(2\theta) &= \cos^2\theta - \sin^2\theta \\
        \tan(2\theta) &= \frac{2\tan\theta}{1 - \tan^2\theta}
    \end{align}

    % \section{Limits}
    % Let \(\const{c}\) be a constant and \(\var<\N>{n} \in \N\)
    % then
    % \begin{align}
    %     \fn{\lim_{\var{x} \to \var{a}}} \const{c}
    %         &\stmt=
    %         \const{c}
    %     \\[1ex]
    %     \fn{\lim_{\var{x} \to \var{a}}} \var{x}
    %         &\stmt=
    %         \var{a}
    %     \\[1ex]
    %     \fn{\lim_{\var{x} \to \var{a}}} \fn{f}[\var{x}]
    %         &\stmt=
    %         \fn{f}[\var{a}]
    %     \\[1ex]
    %     \fn{\lim_{\var{x} \to \var{a}}}[\fn{f}[\var{x}] \op+ \fn{g}[\var{x}]]
    %         &\stmt=
    %         \lim_{\var{x} \to \var{a}} \fn{f}[\var{x}] \op+ \lim_{\var{x} \to \var{a}} \fn{g}[\var{x}]
    %     \\[1ex]
    %     \fn{\lim_{\var{x} \to \var{a}}}[\fn{f}[\var{x}] \op- \fn{g}[\var{x}]]
    %         &\stmt=
    %         \lim_{\var{x} \to \var{a}} \fn{f}[\var{x}] \op- \lim_{\var{x} \to \var{a}} \fn{g}[\var{x}]
    %     \\[1ex]
    %     \fn{\lim_{\var{x} \to \var{a}}}[\const{c} \mul \fn{f}[\var{x}]]
    %         &\stmt=
    %         \const{c} \mul \lim_{\var{x} \to \var{a}} \fn{f}[\var{x}]
    %     \\[1ex]
    %     \fn{\lim_{\var{x} \to \var{a}}}[\fn{f}[\var{x}] \mul \fn{g}[\var{x}]]
    %         &\stmt=
    %         \lim_{\var{x} \to \var{a}} \fn{f}[\var{x}] \mul \lim_{\var{x} \to \var{a}} \fn{g}[\var{x}]
    %     \\[1ex]
    %     \fn{\lim_{\var{x} \to \var{a}}} \br({\frac{\fn{f}[\var{x}]}{\fn{g}[\var{x}]}})
    %         &\stmt=
    %         \frac{\displaystyle\lim_{\var{x} \to \var{a}} \fn{f}[\var{x}]}{\displaystyle\lim_{\var{x} \to \var{a}} \fn{g}[\var{x}]}
    %         \stmt\iff
    %         \lim_{\var{x} \to \var{a}} \fn{g}[\var{x}]
    %         \stmt\ne
    %         \lit{0}
    %     \\[1ex]
    %     \fn{\lim_{\var{x} \to \var{a}}} {\br({\fn{f}[\var{x}]})}^{\var{n}}
    %         &\stmt=
    %         {\br({\fn{\lim_{\var{x} \to \var{a}}} \fn{f}[\var{x}]})}^{\var{n}}
    %     \\[1ex]
    %     \fn{\lim_{\var{x} \to \var{a}}} \op{\sqrt[\var<\N>{n}]{\fn{f}[\var{x}]}}
    %         &\stmt=
    %         \op{\sqrt[\var<\N>{n}]{\fn{\lim_{\var{x} \to \var{a}}} \fn{f}[\var{x}]}}
    % \end{align}

    % \section{Derivatives}
    % % Let \(c\) be a constant, \(f\) and \(g\) be functions, \(n \in \mathbb{R}\), and \(y\) be defined in terms of \(x\).
    % % Then,
    % \begin{align}
    %     \deriv{\var{x}}{\const{c}}
    %         &\stmt=
    %         \lit{0}
    %     \\[1ex]
    %     \deriv{\var{x}}{\var{x}}
    %         &\stmt=
    %         \lit{1}
    %     \\[1ex] % 1x^0
    %     \deriv{\var{x}}{\br({\const{c} \mul \fn{f}[\var{x}]})}
    %         &\stmt=
    %         \const{c} \mul \deriv{\var{x}}{\fn{f}}[\var{x}]
    %     \\[1ex]
    %     \deriv{\var{x}}{\br({\fn{f}[\var{x}] \op\pm \fn{g}[\var{x}]})}
    %         &\stmt=
    %         \deriv{\var{x}}{\fn{f}}[\var{x}] \op\pm \deriv{\var{x}}{\fn{g}}[\var{x}]
    %     \\[1ex]
    %     \deriv{\var{x}}{\br({\fn{f}[\var{x}] \mul \fn{g}[\var{x}]})}
    %         &\stmt=
    %         \deriv{\var{x}}{\fn{f}}[\var{x}] \mul \fn{g}[\var{x}] \op+ \fn{f}[\var{x}] \mul \deriv{\var{x}}{\fn{g}}[\var{x}]
    %     \\[1ex]
    %     \deriv{\var{x}}{\br({\op{\frac{\fn{f}[\var{x}]}{\fn{g}[\var{x}]}}})}
    %         &\stmt=
    %         \op{\frac{
    %             \deriv{\var{x}}{\fn{f}}[\var{x}] \mul \fn{g}[\var{x}] \op- \fn{f}[\var{x}] \mul \deriv{\var{x}}{\fn{g}}[\var{x}]
    %         }{
    %             \fn{g^\lit{2}}[\var{x}]
    %         }}
    %     \\[1ex]
    %     \deriv{\var{x}}{\br({\var{x}^{\var<\N>{n}}})}
    %         &\stmt=
    %         \var<\N>{n} \mul \var{x}^{\var<\N>{n} \op- \lit{1}}
    %     \\[1ex]
    %     \deriv{\var{x}}{\br({\var<\N>{n}^{\var{x}}})}
    %         &\stmt=
    %         \var<\N>{n}^{\var{x}} \mul \fn{\ln}[\var<\N>{n}]
    %     \\[1ex]
    %     \deriv{\var{x}}{\br({\lit{e}^{\var{x}}})}
    %         &\stmt=
    %         \lit{e}^{\var{x}}
    %     \\[1ex]
    %     \deriv{\var{x}}{\fn{\log_{\var<\N>{n}}}}[\var{x}]
    %         &\stmt=
    %         \op{\frac{\lit{1}}{\var{x} \mul \fn{\ln}[\var<\N>{n}]}}
    %     \\[1ex]
    %     \deriv{\var{x}}{{\fn{\ln}}}[\var{x}]
    %         &\stmt=
    %         \op{\frac{\lit{1}}{\var{x}}}
    %     \\[1ex]
    %     \label{chain}
    %     % Options
    %     % - circ
    %     % - enclosed
    %     % - leibniz
    %     \def\ChainRuleSyntax{enclosed}
    %     \ifdefstring{\ChainRuleSyntax}{circ}{
    %         \deriv{\var{x}}{\br({\fn{f} \op\circ \fn{g}})}[\var{x}]
    %             &\stmt=
    %             \fn{\br({\deriv{\var{x}}{\fn{f}} \op\circ \fn{g}})}[\var{x}] \mul \deriv{\var{x}}{\fn{g}}[\var{x}]
    %     }{\ifdefstring{\ChainRuleSyntax}{enclosed}{
    %         \deriv{\var{x}}{\br({\fn{f}[\fn{g}[\var{x}]]})}
    %             &\stmt=
    %             \deriv{\var{x}}{\fn{f}}[\fn{g}[\var{x}]] \mul \deriv{\var{x}}{\fn{g}}[\var{x}]
    %     }{\ifdefstring{\ChainRuleSyntax}{leibniz}{
    %         \fn{\frac{d\var{y}}{d\var{x}}}
    %             &\stmt=
    %             \fn{\frac{d\var{y}}{d\var{u}}}
    %             \mul
    %             \fn{\frac{d\var{u}}{d\var{x}}}
    %     }{
    %         \errmessage{I don't know how to represent ``\ChainRuleSyntax'' chain rule syntax}
    %     }}}
    % \end{align}
    \subsection{Derivatives of Trig Functions}
    \begin{align}
        \deriv{\theta}{\sin}(\theta)
            &\stmt=
            \cos \theta
        \\
        \deriv{\theta}{\cos}(\theta)
            &\stmt=
            -\sin \theta
        \\
        \deriv{\theta}{\tan}(\theta)
            &\stmt=
            \sec^2 \theta
        \\
        \deriv{\theta}{\csc}(\theta)
            &\stmt=
            -\csc(\theta) \cdot \cot(\theta)
        \\
        \deriv{\theta}{\sec}(\theta)
            &\stmt=
            \sec(\theta) \cdot \tan(\theta)
        \\
        \deriv{\theta}{\cot}(\theta)
            &\stmt=
            -\csc^2 \theta
        \\
        \deriv{\theta}{\arcsin}(\theta)
            &\stmt=
            \frac{1}{\sqrt{1-\theta^2}}
        \\
        \deriv{\theta}{\arccos}(\theta)
            &\stmt=
            \frac{-1}{\sqrt{1-\theta^2}}
        \\
        \deriv{\theta}{\arctan}(\theta)
            &\stmt=
            \frac{1}{1+\theta^2}
    \end{align}

    % \section{Integration}
    % \begin{align}
    % \end{align}

    % \chapter{Proofs}

\end{document}
