%%%%%%%%%%%%%%%%%%%%%%%%%%%%% Define Article %%%%%%%%%%%%%%%%%%%%%%%%%%%%%%%%%%
\documentclass{report}
%%%%%%%%%%%%%%%%%%%%%%%%%%%%%%%%%%%%%%%%%%%%%%%%%%%%%%%%%%%%%%%%%%%%%%%%%%%%%%%

%%%%%%%%%%%%%%%%%%%%%%%%%%%%% Using Packages %%%%%%%%%%%%%%%%%%%%%%%%%%%%%%%%%%
\usepackage{geometry}
\usepackage{graphicx}
\usepackage{amssymb}
\usepackage{amsmath}
\usepackage{amsthm}
\usepackage{empheq}
\usepackage{mdframed}
\usepackage{booktabs}
\usepackage{lipsum}
\usepackage{graphicx}
\usepackage{color}
\usepackage{transparent}
\usepackage{psfrag}
\usepackage{pgfplots}
\usepackage{bm}
%%%%%%%%%%%%%%%%%%%%%%%%%%%%%%%%%%%%%%%%%%%%%%%%%%%%%%%%%%%%%%%%%%%%%%%%%%%%%%%

\makeatletter
\DeclareRobustCommand{\lnor}{
    \mathbin{\overline{\vee}}
}
\DeclareRobustCommand{\lxnor}{
    \mathbin{\underline{\wedge}}
}
\makeatother

\def\undefined{\mathrm{undefined}}
\def\dne{\mathrm{DNE}}

\def\lnand{\barwedge}
\def\lxor{\veebar}
\def\true{1} % \top or 1
\def\false{0} % \bot or 0

% Options:
% - leibniz
% - lagrange
% - euler
\newcommand{\CalcNotation}{lagrange}

% \deriv<n>[of]{with respect to}{for the expression}[with the argument(s)]
\NewDocumentCommand{\deriv}{D<>{1}O{}mmO{}}{
    \ifdefstring{\CalcNotation}{leibniz}{
        % Leibniz
        \ifnumequal{#1}{1}{
            \frac{d#2}{d{#3}}{#4#5}
        }{
            \frac{d^{#1}#2}{d{#3}^{#1}}{#4#5}
        }
    }{\ifdefstring{\CalcNotation}{lagrange}{
        % Lagrange (prime)
        \ifnumequal{#1}{1}{
            {#4}'{#5}
        }{\ifnumequal{#1}{2}{
            {#4}''{#5}
        }{
            {#4}^{(#1)}{#5}
        }}
    }{\ifdefstring{\CalcNotation}{euler}{
        % Euler
        \ifnumequal{#1}{1}{
            D_{#3}{#4}{#5}
        }{
            D^{#1}_{#3}{#4}{#5}
        }
    }{
        \errmessage{I don't know how to represent ``\CalcNotation'' notation.}
    }}}
}

% \integ<n>{with respect to}{for the expression}
\NewDocumentCommand{\integ}{D<>{1}mm}{
    \ifdefstring{\CalcNotation}{leibniz}{
        % Leibniz
        \ifnumequal{#1}{1}{
            \int
        }{\ifnumequal{#1}{2}{
            \iint
        }{\ifnumequal{#1}{3}{
            \iiint
        }{\ifnumequal{#1}{4}{
            \iiiint
        }{
            \errmessage{Integral order too high, I dont know how to represent #1-order integrals in Leibniz notation.}
        }}}}{#3}\:{d#2}
    }{\ifdefstring{\CalcNotation}{lagrange}{
        % Lagrange (prime)
        {#3}^{(-#1)}
    }{\ifdefstring{\CalcNotation}{euler}{
        % Euler
        D^{-#1}_{#2}{#3}
    }{
        \errmessage{I don't know how to represent ``\CalcNotation'' notation.}
    }}}
}

\newcounter{brDepth}
\NewDocumentCommand{\br}{mmm}{
    \stepcounter{brDepth}
    \ifnumequal{\value{brDepth}}{4}{\setcounter{brDepth}{1}}{}
    {\color[HTML]{\ifnumequal{\value{brDepth}}{1}{ffd700}{\ifnumequal{\value{brDepth}}{2}{da70d6}{\ifnumequal{\value{brDepth}}{3}{179fff}{\errmessage{unreachable}}}}}\left#1{\color[HTML]{cccccc}#2}\right#3}
    \addtocounter{brDepth}{-1}
    \ifnumequal{\value{brDepth}}{0}{\setcounter{brDepth}{3}}{}
}

\def\R{\type{\mathbb{R}}}
\def\N{\type{\mathbb{N}}}
\def\Z{\type{\mathbb{Z}}}
\def\Const#1{{\color[HTML]{4fc1ff}#1}}
\def\Var#1{{\color[HTML]{9cdcfe}#1}}
\def\Fn#1{{\color[HTML]{dcdcaa}#1}}
\def\type#1{{\color[HTML]{4ec9b0}#1}}
\def\lit#1{{\color[HTML]{b5cea8}#1}}
\def\keyword#1{{\color[HTML]{569cd6}#1}}
\def\op#1{\mathbin{\color[HTML]{569cd6}#1}}
\def\stmt#1{\mathrel{\color[HTML]{c586c0}#1}}
\def\aside#1{\mathrlap{\qquad\color[HTML]{6a9955}#1}}
\def\brA#1{{\color[HTML]{ffd700}#1}}
\def\brB#1{{\color[HTML]{da70d6}#1}}
\def\brC#1{{\color[HTML]{179fff}#1}}
\def\ghost#1{{\transparent{0.34}\color{white}#1}}

% \newcommand{\ShowHints}{}

\def\hint#1#2{
    \ifdef{\ShowHints}{
        #2 {\ghost{\scalebox{0.4}{\ensuremath{#1}}}}
    }{
        #2
    }
}
\NewDocumentCommand{\var}{d<>m}{\IfNoValueTF{#1}{\Var{#2}}{\hint{\;\in\type{#1}}{\Var{#2}}}}
\NewDocumentCommand{\const}{d<>m}{\IfNoValueTF{#1}{\Const{#2}}{\hint{\;\in\type{#1}}{\Const{#2}}}}
\NewDocumentCommand{\fn}{d<>m}{\IfNoValueTF{#1}{\Fn{#2}}{\hint{\to\type{#1}}{\Fn{#2}}}}

%%%%%%%%%%%%%%%%%%%%%%%%%% Page Setting %%%%%%%%%%%%%%%%%%%%%%%%%%%%%%%%%%%%%%%
\geometry{a4paper}
\pagecolor[RGB]{31,31,31}
% \pagecolor[RGB]{0,0,0}
\color[HTML]{cccccc}

%%%%%%%%%%%%%%%%%%%%%%%%%% Define some useful colors %%%%%%%%%%%%%%%%%%%%%%%%%%
\definecolor{ocre}{RGB}{243,102,25}
\definecolor{mygray}{RGB}{243,243,244}
\definecolor{deepGreen}{RGB}{26,111,0}
\definecolor{shallowGreen}{RGB}{235,255,255}
\definecolor{deepBlue}{RGB}{61,124,222}
\definecolor{shallowBlue}{RGB}{235,249,255}
%%%%%%%%%%%%%%%%%%%%%%%%%%%%%%%%%%%%%%%%%%%%%%%%%%%%%%%%%%%%%%%%%%%%%%%%%%%%%%%

%%%%%%%%%%%%%%%%%%%%%%%%%% Define an orangebox command %%%%%%%%%%%%%%%%%%%%%%%%
\newcommand\orangebox[1]{\fcolorbox{ocre}{mygray}{\hspace{1em}#1\hspace{1em}}}
%%%%%%%%%%%%%%%%%%%%%%%%%%%%%%%%%%%%%%%%%%%%%%%%%%%%%%%%%%%%%%%%%%%%%%%%%%%%%%%

%%%%%%%%%%%%%%%%%%%%%%%%%%%% English Environments %%%%%%%%%%%%%%%%%%%%%%%%%%%%%
\newtheoremstyle{mytheoremstyle}{3pt}{3pt}{\normalfont}{0cm}{\rmfamily\bfseries}{}{1em}{{\color{black}\thmname{#1}~\thmnumber{#2}}\thmnote{\,--\,#3}}
\newtheoremstyle{myproblemstyle}{3pt}{3pt}{\normalfont}{0cm}{\rmfamily\bfseries}{}{1em}{{\color{black}\thmname{#1}~\thmnumber{#2}}\thmnote{\,--\,#3}}
\theoremstyle{mytheoremstyle}
\newmdtheoremenv[linewidth=1pt,backgroundcolor=shallowGreen,linecolor=deepGreen,leftmargin=0pt,innerleftmargin=20pt,innerrightmargin=20pt,]{theorem}{Theorem}[section]
\theoremstyle{mytheoremstyle}
\newmdtheoremenv[linewidth=1pt,backgroundcolor=shallowBlue,linecolor=deepBlue,leftmargin=0pt,innerleftmargin=20pt,innerrightmargin=20pt,]{definition}{Definition}[section]
\theoremstyle{myproblemstyle}
\newmdtheoremenv[linecolor=black,leftmargin=0pt,innerleftmargin=10pt,innerrightmargin=10pt,]{problem}{Problem}[section]
%%%%%%%%%%%%%%%%%%%%%%%%%%%%%%%%%%%%%%%%%%%%%%%%%%%%%%%%%%%%%%%%%%%%%%%%%%%%%%%

%%%%%%%%%%%%%%%%%%%%%%%%%%%%%%% Plotting Settings %%%%%%%%%%%%%%%%%%%%%%%%%%%%%
\usepgfplotslibrary{colorbrewer}
\pgfplotsset{width=8cm,compat=1.9}
%%%%%%%%%%%%%%%%%%%%%%%%%%%%%%%%%%%%%%%%%%%%%%%%%%%%%%%%%%%%%%%%%%%%%%%%%%%%%%%

%%%%%%%%%%%%%%%%%%%%%%%%%%%%%%% Title & Author %%%%%%%%%%%%%%%%%%%%%%%%%%%%%%%%
\title{Math Properties}
\author{Amy Wilder}
%%%%%%%%%%%%%%%%%%%%%%%%%%%%%%%%%%%%%%%%%%%%%%%%%%%%%%%%%%%%%%%%%%%%%%%%%%%%%%%

\begin{document}
    \maketitle

    \chapter{Properties}

    Assume the following:
    \begin{itemize}
        \item \(\const{h},\const{k}\) are constants
        \item \(\var{\theta},\var{\phi},\var{\psi}\) are angles (the golden ratio is \(\lit{\varphi}\))
        \item \(\var{a},\var{b},\var{c},\var{x} \in \R\)
        \item \(\var{y}\) is defined in terms of \(\var{x}\)
        \item \(\fn{f}\) and \(\fn{g}\) are functions
        \item \(\var<\N>{n},\var<\N>{m} \in \N\)
    \end{itemize}

    \section{Reflexive}
    \begin{gather}
        \var{x} \stmt= \var{x} \\
        \var{a} \stmt< \var{b} \stmt{\iff} \var{b} \stmt> \var{a}
    \end{gather}

    \section{Mutation}
    \begin{align}
        \var{x} \op+ \lit{1} &\stmt> \var{x} \\
        \var{x} \op- \lit{1} &\stmt< \var{x} \\
        \op{ |\var{x} \op\cdot \lit{2}| } &\stmt> \var{x}
    \end{align}

    \section{Positivity}
    \begin{align}
        \op{ |\var{x}| } &\stmt\ge \lit{0} \\
        \var{x}^{\lit{2} \op\cdot \var<\N>{n}} &\stmt\ge \lit{0}
    \end{align}

    \section{Identity}
    \begin{align}
        \var{x} \op+ \lit{0} &\stmt= \var{x} \\
        \var{x} \op- \lit{0} &\stmt= \var{x} \\
        \var{x} \op\cdot \lit{1} &\stmt= \var{x} \\
        \op{\frac{\var{x}}{\lit{1}}} &\stmt= \var{x} \\
        \var{x}^\lit{1} &\stmt= \var{x} \\
        \fn{\deriv{\var{x}}{\brA{(\lit{e}^{\var{x}})}}} &\stmt= \lit{e}^{\var{x}}
    \end{align}

    \section{Fixed Point}
    \begin{align}
        \var{x} \op\cdot \lit{0} &\stmt= \lit{0} \\
        {\var{x}}^\lit{0} &\stmt= \lit{1} \\
        \fn{\log_{\var{x}} \lit{1}} &\stmt= \lit{0}
    \end{align}

    \section{Self-Destructive}
    \begin{align}
        \var{x} \op- \var{x} &\stmt= \lit{0} \\
        \op{\frac{\var{x}}{\var{x}}} &\stmt= \lit{1} \aside{\iff \var{x} \stmt\ne \lit{0}} \\
        \fn{\log_{\var{x}} \var{x}} &\stmt= \lit{1} \\
        \fn{\ln \lit{e}} &\stmt= \lit{1}
    \end{align}

    \section{Commutative}
    \begin{align}
        \var{a} \op+ \var{b} &\stmt= \var{b} \op+ \var{a} \\
        \var{a} \op\cdot \var{b} &\stmt= \var{b} \op\cdot \var{a}
    \end{align}

    \section{Associative}
    \begin{align}
        \brA{(\var{a} \op+ \var{b})} \op+ \var{c} &\stmt= \var{a} \op+ \brA{(\var{b} \op+ \var{c})} \\
        \brA{(\var{a} \op\cdot \var{b})} \op\cdot \var{c} &\stmt= \var{a} \op\cdot \brA{(\var{b} \op\cdot \var{c})}
    \end{align}

    \section{Distributive}

    %%%%%%%%%%%%%%%%%%%%%%%%%%%%%%%%%%%%%%%%%%%%%%%%%%%%%%

    \chapter{Operations}

    \section{Arithmetic}
    \begin{align}
        a + b &= b + a \label{eq:commut-add} \\
        a \cdot b &= b \cdot a \label{eq:commut-mult} \\
        (a + b) + c &= a + (b + c) = a + b + c \label{eq:assoc-add} \\
        (a \cdot b) \cdot c &= a \cdot (b \cdot c) = a \cdot b \cdot c \label{eq:assoc-mul} \\
        a \cdot (b + c) &= a \cdot b + a \cdot c \label{distrib-mul} \\
        \frac{a + b}{c} &= \frac{a}{c} + \frac{b}{c} \label{distrib-div}
    \end{align}

    \subsection{Logarithms}
    \begin{align*}
        \log_a b = y &\iff a^y = b \\
        \log x &= \log_{10} x \\
        \ln x &= \log_{e} x \\
    \end{align*}
    \begin{align}
        \log_n\left( a \cdot b \right) &= \log_n a + \log_n b \\
        \log_n\left( \frac{a}{b} \right) &= \log_n a - \log_n b \\
        \log_n\left( a^b \right) &= b \cdot \log_n a\\
        \log_n 1 &= 0
    \end{align}

    \subsubsection{Natural}
    \begin{align}
        \ln\left( a \cdot b \right) &= \ln a + \ln b \\
        \ln\left( \frac{a}{b} \right) &= \ln a - \ln b \\
        \ln\left( a^b \right) &= b\ln a \\
        \ln e &= 1
    \end{align}

    \section{Boolean}
    \subsection{Definitions}
    \begin{align*}
        \lnot  &= \text{NOT}  \\
        \land  &= \text{AND}  \\
        \lnand &= \text{NAND} \\
        \lor   &= \text{OR}   \\
        \lnor  &= \text{NOR}  \\
        \lxor  &= \text{XOR}  \\
        \lxnor &= \text{XNOR}
    \end{align*}
    \subsection{Using only NOT and OR}
    \begin{align}
        a \lor b &= a \lor b \\
        a \lnor b &= \lnot(a \lor b) \\
        a \land b &= \lnot(\lnot a \lor \lnot b) \\
        a \lnand b &= \lnot a \lor \lnot b \\
        a \lxor b &= \lnot(\lnot (a \lor b) \lor \lnot(\lnot a \lor \lnot b)) \\
        a \lxnor b &= \lnot(a \lor b) \lor \lnot (\lnot a \lor \lnot b)
    \end{align}
    \subsection{Commutative}
    \begin{align}
        a \lor b &= b \lor a \\
        a \lnor b &= b \lnor a \\
        a \land b &= b \land a \\
        a \lnand b &= b \lnand a \\
        a \lxor b &= b \lxor a \\
        a \lxnor b &= b \lxnor a \\
    \end{align}
    \subsection{Associative}
    \begin{align}
        (a \land b) \land c &= a \land (b \land c) = a \land b \land c \\
        (a \lor b) \lor c &= a \lor (b \lor c) = a \lor b \lor c \\
    \end{align}
    \begin{align}
        % negation
        \lnot(\lnot x) &= x \\
        \lnot(a \land b) &= (\lnot a \lor \lnot b) \\
        \lnot(a \lor b) &= (\lnot a \land \lnot b) \\
        (a \land b) \lor (a \land c) &= a \land (b \lor c) \\
        (a \lor b) \land (a \lor c) &= a \lor (b \land c) \\
        x \land x &= x \\
        x \lor x &= x \\
        x \land \true &= x \\
        x \lor \false &= x \\
        x \land \false &= \false \\
        x \lor \true &= \true
    \end{align}

    \section{Trig}
    \subsection{Definitions}
    \begin{align}
        \sin\theta &= \frac{y}{r} = \frac{1}{\csc\theta} \\
        \cos\theta &= \frac{x}{r} = \frac{1}{\sec\theta} \\
        \tan\theta &= \frac{y}{x} = \frac{1}{\cot\theta} = \frac{\sin\theta}{\cos\theta} \\
        \csc\theta &= \frac{r}{y} = \frac{1}{\sin\theta} \\
        \sec\theta &= \frac{r}{x} = \frac{1}{\cos\theta} \\
        \cot\theta &= \frac{x}{y} = \frac{1}{\tan\theta} = \frac{\cos\theta}{\sin\theta}
    \end{align}
    \subsection{Pythagorean}
    \begin{align}
        \cos^2\theta+\sin^2\theta &= 1 \\
        \sin^2\theta &= 1 - \cos^2 \theta \\
        \sin\theta &= \pm\sqrt{1 - \cos^2 \theta} \\
        \cos^2\theta &= 1 - \sin^2 \theta \\
        \cos\theta &= \pm\sqrt{1 - \sin^2 \theta} \\
    \end{align}
    \subsection{Angle Sum and Difference}
    \begin{align}
        \sin(\alpha \pm \beta) &= \sin\alpha\cdot\cos\beta \pm \cos\alpha\cdot\sin\beta \\
        \cos(\alpha \pm \beta) &= \cos\alpha\cdot\cos\beta \mp \sin\alpha\cdot\sin\beta \\
        \tan(\alpha \pm \beta) &= \frac{\tan\alpha \pm \tan\beta}{1 \mp \tan\alpha\cdot\tan\beta} \\
        \sin(2\theta) &= 2\sin\theta\cdot\cos\theta \\
        \cos(2\theta) &= \cos^2\theta - \sin^2\theta \\
        \tan(2\theta) &= \frac{2\tan\theta}{1 - \tan^2\theta}
    \end{align}

    \section{Limits}
    Let \(c\) be a constant and \(n \in \mathbb{N}\)
    then
    \begin{align}
        \lim_{x \to a}c &= c \\
        \lim_{x \to a}x &= a \\
        \lim_{x \to a}f(x) &= f(a) \\
        \lim_{x \to a}\bigl(f(x) + g(x)\bigr) &= \lim_{x \to a} f(x) + \lim_{x \to a} g(x) \\
        \lim_{x \to a}\bigl(f(x) - g(x)\bigr) &= \lim_{x \to a} f(x) - \lim_{x \to a} g(x) \\
        \lim_{x \to a}\bigl(c \cdot f(x)\bigr) &= c \cdot \lim_{x \to a} f(x) \\
        \lim_{x \to a}\bigl(f(x) \cdot g(x)\bigr) &= \lim_{x \to a} f(x) \cdot \lim_{x \to a} g(x) \\
        \lim_{x \to a}\left(\frac{f(x)}{g(x)}\right) &= \frac{\displaystyle\lim_{x \to a} f(x)}{\displaystyle\lim_{x \to a} g(x)} \iff \lim_{x \to a} g(x) \ne 0 \\
        \lim_{x \to a}{\bigl(f(x)\bigr)}^n &= {\left(\lim_{x \to a} f(x)\right)}^n \\
        \lim_{x \to a}\sqrt[n]{f(x)} &= \sqrt[n]{\lim_{x \to a} f(x)}
    \end{align}

    \section{Derivatives}
    % Let \(c\) be a constant, \(f\) and \(g\) be functions, \(n \in \mathbb{R}\), and \(y\) be defined in terms of \(x\).
    % Then,
    \begin{align}
        \fn{\deriv{\var{x}}{\const{c}}}
            &\stmt=
            \lit{0}
        \\
        \fn{\deriv{\var{x}}{\var{x}}}
            &\stmt=
            \lit{1}
        \\ % 1x^0
        \fn{\deriv{\var{x}}{\br({\const{c} \op\cdot \fn{f}\br({\var{x}})})}}
            &\stmt=
            \const{c} \op\cdot \fn{\deriv{\var{x}}{\fn{f}}[\br({\var{x}})]}
        \\
        \fn{\deriv{\var{x}}{\br({\fn{f}\br({\var{x}}) \op\pm \fn{g}\br({\var{x}})})}}
            &\stmt=
            \fn{\deriv{\var{x}}{\fn{f}}[\br({\var{x}})]} \op\pm \fn{\deriv{\var{x}}{\fn{g}}[\br({\var{x}})]}
        \\
        \fn{\deriv{\var{x}}{\br({\fn{f}\br({\var{x}}) \op\cdot \fn{g}\br({\var{x}})})}}
            &\stmt=
            \fn{\deriv{\var{x}}{\fn{f}}[\br({\var{x}})]} \op\cdot \fn{g}\br({\var{x}}) \op+ \fn{f}\br({\var{x}}) \op\cdot \fn{\deriv{\var{x}}{\fn{g}}[\br({\var{x}})]}
        \\
        \fn{\deriv{\var{x}}{\br({\op{\frac{\fn{f}\br({\var{x}})}{\fn{g}\br({\var{x}})}}})}}
            &\stmt=
            \op{\frac{
                \fn{\deriv{\var{x}}{\fn{f}}[\br({\var{x}})]} \op\cdot \fn{g}\br({\var{x}}) \op- \fn{f}\br({\var{x}}) \op\cdot \fn{\deriv{\var{x}}{\fn{g}}[\br({\var{x}})]}
            }{
                \fn{g}^\lit{2}\br({\var{x}})
            }}
        \\
        \fn{\deriv{\var{x}}{\br({\var{x}^{\var<\N>{n}}})}}
            &\stmt=
            \var<\N>{n} \op\cdot \var{x}^{\var<\N>{n} \op- \lit{1}}
        \\
        \fn{\deriv{\var{x}}{\br({\var<\N>{n}^{\var{x}}})}}
            &\stmt=
            \var<\N>{n}^{\var{x}} \op\cdot \fn{\ln\br({\var<\N>{n}})}
        \\
        \fn{\deriv{\var{x}}{\br({\lit{e}^{\var{x}}})}}
            &\stmt=
            \lit{e}^{\var{x}}
        \\
        \fn{\deriv{\var{x}}{\fn{\log_{\var<\N>{n}}}}[\br({\var{x}})]}
            &\stmt=
            \op{\frac{\lit{1}}{\var{x} \op\cdot \fn{\ln\br({\var<\N>{n}})}}}
        \\
        \fn{\deriv{\var{x}}{{\fn{\ln}}}[\br({\var{x}})]}
            &\stmt=
            \op{\frac{\lit{1}}{\var{x}}}
        \\
        \fn{\deriv{\var{x}}{\br({\fn{f} \op\circ \fn{g}})}\br({\var{x}})}
            &\stmt=
            \br({\fn{\deriv{\var{x}}{\fn{f}}} \op\circ \fn{g}})\br({\var{x}}) \op\cdot \fn{\deriv{\var{x}}{\fn{g}}[\br({\var{x}})]}
            \label{chain}
        \\
        \fn{\deriv{\var{x}}{\br({\var{y}^\lit{2}})}}
            &\stmt=
            \lit{2} \op\cdot \var{y} \op\cdot \fn{\deriv{\var{x}}{\var{y}}} \label{implicit}
    \end{align}
    \subsection{Derivatives of Trig Functions}
    \begin{align}
        \deriv{\theta}{\sin}[\theta]
            &\stmt=
            \cos \theta
        \\
        \deriv{\theta}{\cos}[\theta]
            &\stmt=
            -\sin \theta
        \\
        \deriv{\theta}{\tan}[\theta]
            &\stmt=
            \sec^2 \theta
        \\
        \deriv{\theta}{\csc}[\theta]
            &\stmt=
            -\csc(\theta) \cdot \cot(\theta)
        \\
        \deriv{\theta}{\sec}[\theta]
            &\stmt=
            \sec(\theta) \cdot \tan(\theta)
        \\
        \deriv{\theta}{\cot}[\theta]
            &\stmt=
            -\csc^2 \theta
        \\
        \deriv{\theta}{\arcsin}[\theta]
            &\stmt=
            \frac{1}{\sqrt{1-\theta^2}}
        \\
        \deriv{\theta}{\arccos}[\theta]
            &\stmt=
            \frac{-1}{\sqrt{1-\theta^2}}
        \\
        \deriv{\theta}{\arctan}[\theta]
            &\stmt=
            \frac{1}{1+\theta^2}
    \end{align}

    \section{Integration}
    \begin{align}
    \end{align}

    \chapter{Proofs}

\end{document}
